\section{Evaluation der Methode an Beispielen}

In diesem Kapitel soll die Anwendbarkeit und Wirksamkeit von \ac{MFEM} untersucht werden. Hierzu wird die Methode an 2 Beispielen durchgeführt und ausgewertet. Es werden sämtliche Phase exemplarisch durchlaufen und dokumentiert. Zudem wird für die Durchführung der 2 Beispiele der Zeitaufwand gemessen, um auch eine Aussage über die Anwendungsdauer zu treffen. Diese spielt eine Rolle für die Akzeptanz der Methode, da eine lange Laufzeit eventuelle Nutzer abschrecken könnte.\\ 
Als Ergebnis der Evaluation wird sich zeigen, ob die Methode möglichst viele Seiten eines Frameworks betrachtet und dabei repräsentative Resultate liefert.  
Um dies zu gewährleisten, müssen die richtigen Kandidaten gewählt werden.

\subsection{Wahl der Kandidaten}

\ac{MFEM} sollte möglichst ein schlechte Frameworks negativ und Gute positiv bewerten. Aus diesem Grund wurde als Erstes das Spring Framework\footnote{\url{spring.io}} gewählt. Es befindet sich seit 2002 in Entwicklung und hat seit dem schon mehrere Preise gewonnen\cite{Gutierrez2016}. Dabei entwickelt es die sehr aktive Open-Source-Community ständig weiter und integriert die neuesten Technologien. Beispiele hierfür sind WebFlux, welches reaktive Webapplikationen ermöglicht, oder Cloud-Contract, um die \ac{REST}-Schnittstelle mit einem Consumer-Driven Contract abzusichern.\\
Um dem Entwickler so viel Arbeit wie möglich abzunehmen, sodass er sich auf die Geschäftslogik konzentrieren kann, wurde das Projekt Spring Boot aufgenommen. Es stellt die Konvention über die Konfiguration und erstellt so automatisch robuste, erweiterbare und skalierbare Spring Applikationen\cite[1]{Gutierrez2016}.

\subsection{Kickoff-Phase}
\subsection{Analysephase}
\subsubsection{Wahl der Anforderungen}
\subsubsection{Metriken definieren}
\subsection{Evaluationsphase}
\subsubsection{Evaluation definieren}
\subsubsection{Metriken zuordnen}
\subsubsection{Evaluation durchführen: Spring Boot}
\subsubsection{Evaluation durchführen: Go-Kit}
\subsection{Abschlussphase}
\subsection{Methodenauswertung}

