\section{Microservices}

\bigquote{Do One Thing and Do It Well}{Douglas McIlroy (A Quarter Century of Unix)}

Diesem Grundsatz aus der Unix Philosophie folgt auch die Microservice Architektur. Sie bietet einen Ansatz zur Modularisierung von Software. Doch im Gegensatz zur konventionellen Modularisierung von großen Systemen, sind die sogenannten Microservices eigene Programme\cite[10]{Wolff2015}, die über eine einheitliche Schnittstelle, wie z.~B. \ac{REST} oder Messaging, kommunizieren. Für sich genommen ist ein einzelner Service relativ klein und bildet einen mehr oder minder geschlossen Teil der Anwendung ab\cite{Roewekamp2016}.\\
Doch ist dies nur ein Aspekt des Entwurfsmusters. Die Abbildung \ref{MicroservicesVorteile} zeigt hierzu die wesentlichen Vorteile einer Microservice Architektur.

\image[MicroservicesVorteile.pdf]{MicroservicesVorteile}{Vorteile Microservices}{Wesentliche Vorteile von Microservices\cite[12]{Wolff2015}}

\begin{description}[leftmargin=!,labelwidth=\widthof{\bfseries Ergänzung Legacy-Systeme}]
	\item[Starke Modularisierung] 
	 	Durch die strikte Trennung der einzelnen Services in eigene Programme, schleichen sich kaum unerwünschte Abhängigkeiten unter den Modulen ein.
	\item[Ersetzbarkeit]
		Solange die Schnittstelle eines Services gleich bleibt, kann dieser einfach ersetzt werden. Dabei muss er weder die Code-Basis noch die Technologie des bestehenden Services verwenden.
	\item[Nachhaltige Entwicklung] 
		Mit der starken Modularisierung und der Ersetzbarkeit können technische Schulden vermieden und so eine nachhaltige Entwicklung geschaffen werden.
	\item[Effiziente Skalierung]
		Durch die starke Trennung lassen sich die einzelnen Module unabhängig voneinander skalieren.
	\item[Continuous Delivery]
		Kleine Services können schneller in die Produktion gebracht werden, wodurch sich die Time-To-Market Zeit verringert. 
	\item[Technologie Freiheit]
		Einzelne Services können unterschiedliche Technologien verwenden. Es besteht dabei keine Einschränkung auf eine Programmiersprache oder Plattform.
\end{description}

Besonders die technologische Freiheit ist an dieser Stelle interessant. Durch die Aufteilung in einzelnen Services, können für jeden Microservice unabhängig die Programmiersprache und Frameworks gewählt werden. So lassen sich neue Technologien in einem Service erproben, ohne dass andere Services davon betroffen sind. Dies senkt das Risiko für die Einführung neuer Technologien und die Kosten bleiben kalkulierbar\cite[13]{Wolff2015}, da nur in einem kleinen Rahmen eingeführt und getestet wird.\\ 
Aber auch die Fähigkeiten des Entwickler-Teams spielen eine Rolle und können bei der Wahl des Frameworks berücksichtigt werden. Als Beispiel kann das eine Team einen Service mit NodeJS entwickeln, da überwiegend Java\-script Entwickler vertreten sind. Und das andere Team einen Service mit Spring, da dieses hauptsächlich Java beherrschen.

\subsection{Microservice Frameworks}\label{Microservice_Frameworks}

Ein Microservice Framework soll die Erstellung eines Services vereinfachen. Es unterstützt den Entwickler bei sich wiederholenden Tätigkeiten, wie z.~B. der Erstellung eines \ac{REST}-Endpunktes, und bringt viele häufig benötigte Funktionen mit. Dies könnte z.~B. eine Authentifizierung oder Caching sein.\\
Dabei ist es darauf ausgelegt, möglichst schnell Ergebnisse zu erzielen und einen lauffähigen Microservice zu erstellen. Dies ist wichtig, da aufgrund der Vielzahl von Services die Erstellung eines einzigen nicht zu komplex sein sollte und der Fokus auf der Geschäftslogik liegt. 

Dank der Technologiefreiheit kann für jeden Service ein eigenes Framework gewählt werden. Dabei stellt sich die Frage, welches für die Lösung der Teilaufgabe am besten geeignet ist und den Entwickler möglichst gut unterstützt.\\
Da es einen Rahmen für die Entwicklung zur Verfügung stellt und die Programmiersprache vorgibt, ist diese Entscheidung maßgeblich für den weiteren Entwurf des Services. Aus diesem Grund sollte die Wahl nicht gänzlich unbedacht sein. Auch wen mit der Ersetzbarkeit die Hürde zum Austausch gering ist, bedeutet ein Fehlschlag mitunter mehrere Wochen vergeudetet Entwicklungszeit.

Um das richtige Framework zu finden, müssen Anforderungen an dieses gestellt werden. Da ein Microservice eine eigene Anwendung ist und das Framework maßgeblich dazu beiträgt diesen zu erstellen, ergeben sich die Anforderungen aus denen, die an den Service gestellt werden. Diese Serviceanforderungen definieren nicht nur was für eine Aufgabe er hat, sondern auch wie er sie zu erfüllen hat.\\ 
Wobei für das Framework nur die Serviceanforderungen interessant sind, die nicht mit der Geschäftslogik zusammenhängen. Es soll schließlich nur den Rahmen dazu anbieten und den Entwickler weitestgehend bei der Umsetzung der Aufgabe helfen.\\
Welche Anforderungen an einen Service gestellt werden, hängt von der Architektur und dessen Anforderungen ab.
Dabei müssen sie nicht unbedingt direkt aus Anforderungen des Gesamtsystems entstehen. Einzelne Architekturentscheidungen können auch zu Serviceanforderungen führen. Beispielhaft sind hier Service-Discovery und Logging zu nennen. 
Abbildung \ref{Serviceanforderungen} ist ein Versuch diesen komplexen Zusammenhang darzustellen.\\


\image[Serviceanforderungen.pdf][width=0.5\linewidth]{Serviceanforderungen}{Serviceanforderungen}{Aus Anforderungen an die gesamte Architektur ergeben sich separate Serviceanforderungen. Diese können unmittelbar übergehen oder aus Architekturentscheidungen entstehen.}

Im Gegensatz zu einer Bibliothek beeinflusst ein Framework die Servicearchitektur erheblich. Der Entwickler bekommt mit ihr eine Struktur für die Anwendung vorgegeben und erweitert bzw. passt sie so an, dass die Serviceanforderungen erfüllt werden.\\
Damit dies schon während der Entwurfsphase überprüft werden kann, gibt es etablierte Methoden zur Qualitätsbewertung von Softwarearchitekturen. So kann die Servicearchitektur auf die Serviceanforderungen untersucht werden. Für die Bewertung von Frameworks ist es daher naheliegend diese daran anzulehnen.  
 
\pagebreak
\section{Qualitätsbewertung von Softwarearchitektur}\label{Qualitätsbewertung_Softwarearchitektur}

\bigquote{You cannot control what you cannot measure}{Tom DeMarco (Controlling Software Projects)}

Die Aussage beschreibt sehr gut, dass ein IT-Projekt besser auf die Projektziele
zusteuern kann, wenn Messwerte erhoben werden. Diese lassen sich z.~B. in Diagrammen sehr gut darstellen und können zumeist einfach sowie unmissverständlich interpretiert werden. Neben dem aktuellen Projektstatus lassen sich so auch Trends erkennen. Diese helfen dabei rechtzeitig Abweichungen festzustellen und gegebenenfalls korrigierende Maßnahmen einzuleiten\cite{Starke2015}. Auch wenn \citeauthor{DeMarco2009} \citeyear{DeMarco2009} seine Aussage relativiert hat\cite{DeMarco2009}, bleibt sie in Bezug auf die Softwarearchitektur stimmig.
Gerade bei einem so entscheidendem Punkt, wie der Architektur, ist es wichtig frühzeitig
Risiken und Qualitätsabweichungen festzustellen. Die Architektur ist ein sehr kritischer
und wesentlicher Bestandteil im Entwicklungsprozess von einer Software. Ihrer Beschaffenheit nach,
kann sie nur schwer und mit hohen Kosten verändert werden.\\
Durch Zeit- und Kostendruck wird dies leider in der Praxis häufig erst in späten Entwicklungsphasen durchgeführt. 
In einem sogenannten \enquote{Audit} oder \enquote{Review} werden bereits produktiv laufende Systeme bewertet\cite{Starke2015}.
Sollten sich hier starke Abweichungen gegenüber den Anforderungen zeigen, kann sich dies zu einem großen Problem 
avancieren, hohe Kosten verursachen und Kunden verärgern.\\
Aus diesem Grund ist die Qualität einer Architektur sehr entscheidend und sollte stichhaltig nachgewiesen sein.
Um sie zu bestimmen und Restrisiken minimieren zu können, gibt es Methoden zur Qualitätsbewertung 
von Softwarearchitektur. Neben diesem Ziel gibt es noch weitere positive Auswirkungen, die für den Einsatz der Architekturbewertung sprechen. (Siehe Bild \ref{Architekturbewertung})

\image[Architekturbewertung.pdf][width=0.7\linewidth]{Architekturbewertung}{Ziele Architekturbewertung}{Allgemeine Ziele von Architekturbewertung.}

Die Bewertung kann sich dabei auf die im Softwareprojekt entstehenden Artefakte stützen. Beispielsweise sind
Anforderungen, Diagramme, Quellcode und andere Dokumente zu nennen. Diese Artefakte können dabei quantitativ 
oder qualitativ bewertet werden. So kann z.~B. der Quellcode mittels Metriken quantitativ, d.~h. in reinen Zahlen, bewertet werden. Dies lässt sich meist einfach und automatisiert, siehe SonarQube\footnote{\url{https://www.sonarqube.org}}, erfassen und kann sehr gut reproduziert sowie verglichen werden. Der Einsatz sollte jedoch gut überlegt sein, damit nicht ziellos gemessen wird.
Andere Artefakte entziehen sich dieser Bewertung und können nur auf ihre Güte hin, also qualitativ, bewertet werden. Diese, teils auch subjektiven, Bewertungen setzen einen größeren Aufwand voraus und lassen sich schwer vergleichen.\\  
Letzteres macht dabei den größten Teil der Bewertung von Softwarearchitektur aus, da sich anhand der Diagramme und Dokumente nur wenig quantitativ erfassen lässt.     

\subsection{Qualitätsmerkmale der Softwarearchitektur}

Bei der Architekturbewertung sind die Anforderungen das Fundament. Sie charakterisieren die 
gewünschten Eigenschaften eines Systems\cite{Starke2015}.
Bei der Softwarearchitektur gibt es eine Vielzahl an Qualitätsmerkmale anhand derer Anforderungen bestimmt und definiert werden können\cites{Starke2015}{Clements2000}:

\begin{description}[leftmargin=!,labelwidth=\widthof{\bfseries Konzeptuelle Integrität}]
	\item[Funktionalität] 
	Mit der Funktionalität wird bestimmt, ob die funktionalen Anforderungen nicht nur vollständig erfüllt werden, sondern auch richtig und angemessen umgesetzt sind.
	\item[Zuverlässigkeit] 
	Aus der Zuverlässigkeit ergibt sich die Fähigkeit eines Systems, die Funktion innerhalb einer Zeitspanne umzusetzen und
	dabei auf etwaige Fehler zu reagieren bzw. diese zu verhindern.
	\item[Leistung] 
	Die Leistung (Performance) gibt an, wie schnell das System reagiert und wie viele Eingaben pro Zeiteinheit verarbeitet werden können. Diese Anforderungen
	lassen sich meist über Benchmarks testen.
	\item[Flexibilität] 
	Eine Architektur ist flexibel, wenn sie angepasst und erweitert werden kann, ohne dabei das gesamte Konstrukt zu zerstören.
	\item[Übertragbarkeit] 
	Durch die Übertragbarkeit lässt sich das System auch unter verschiedensten Voraussetzungen betreiben. (Hard- und Softwareumgebungen) 
	\item[Unterteilbarkeit]
	Die Unterteilbarkeit ermöglicht eine einfache Aufteilung in Teilsysteme. So können einzelne Teile unabhängig entwickelt oder sogar betrieben werden.
	\item[Konzeptuelle Integrität] 
	Das Konzept sollte sich auf allen Ebenen widerspiegeln und sich durch ein einheitliches Design präsentieren. Ähnliche Dinge sollen dabei in ähnlicher Art und Weise gelöst werden. 
	\item[Machbarkeit] 
	Es muss möglich sein, das System mit vorhandenen Ressourcen (Technologie, Budget, usw.) umzusetzen.
\end{description}

Diese Aufzählung stellt hierbei eine Auswahl an verschiedenen allgemeinen Qualitätsmerkmalen dar und erhebt dabei keinen Anspruch auf Vollständigkeit.

\subsection{Soll-Ist-Vergleich}
Mit den Anforderungen kann ein Soll-Ist-Vergleich anhand der Artefakte durchgeführt werden. Jede einzelne Anforderung wird dabei mit Plänen, Dokumentation oder Modellen verglichen und bewertet. D.~h. es wird geprüft, ob das geplante System den Anforderungen entsprechen wird. Wichtig ist, dass die geforderten Eigenschaften möglichst feingranular sind. Je detaillierter die Anforderungen vorliegen, desto geringer ist das Restrisiko einzelne Punkte zu vergessen oder unscharf zu bewerten.
  
Das Ergebnis dieser Prüfung kann folgendermaßen aussehen\cite{Starke2015}:

\begin{description}[leftmargin=!,labelwidth=\widthof{\bfseries Soll $=$ Is}]
	\item[Soll $=$ Ist] 
	Das Soll wird erfüllt und die Architektur besitzt alle geforderten Eigenschaften.
	\item[Soll $\approx$ Ist] 
	Das Soll wird nur teilweise erreicht und es werden Kompromisse geschlossen. (Verbesserung einzelner Anforderungen)
	\item[Soll $\neq$ Ist] 
	Das Soll wird nicht erfüllt und es ergibt sich ein Risiko für das System
\end{description}

Die Architekturbewertung ist somit ein relativer Vergleich in Hinblick auf spezifische 
Kriterien und liefert keine absolute Aussage über die Qualität. Vielmehr identifizieren sich aus einer Architekturbewertung Risiken, die Architekturentscheidungen in der Entwurfsphase mit sich gebracht haben. 

Diesen Ansatz verfolgt auch \ac{SAAM}. Das \ac{SEI}, eine US-Bundeseinrichtung für die Verbesserung von Software-Engineering Praktiken, hat mit dieser Methode den Grundstein der Architekturbewertung gelegt.
Viele der heute bekannten 
Methoden basieren auf der \ac{SAAM} und erweitern diese um spezielle Sichten oder fokussieren sich auf ein spezielles Qualitätsmerkmal. Mit der \ac{ATAM} wurde die \ac{SAAM} von der \ac{SEI} weiterentwickelt und stellt heute die führende Methode zur Architekturbewertung dar\cite{ATAM_SEI}.

\subsection{\acf*{ATAM}}

Das Ziel von \ac{ATAM} ist nicht eine exakte Vorhersage über die 
zu erwartende Qualität der Software zu treffen. Das ist nahezu unmöglich bei der frühen Entwurfsphase, da noch nicht genügend Informationen 
vorliegen. Vielmehr soll der Blick auf einzelne Qualitätsmerkmale und dessen Abhängigkeit zu gewissen Architektur-Entwurfsentscheidungen geschärft 
werden\cite{Clements2000}. Mit diesem Wissen können einzelne Entscheidungen überdacht, genauer modelliert oder angepasst werden, um die gewünschten 
Qualitätsziele zu erreichen. Darüber hinaus wird auch die Dokumentation der Architektur verbessert, da alle Qualitätsaspekte genauer untersucht werden.

Das Ziel von \ac{ATAM} ist eine Dokumentation von Risiken (risks), Sensitivitätspunkte (sensitivity points) und Kompromisspunkte(tradeoff points), die durch eine genauere Analyse der Architektur\cite{Clements2000} ermittelt werden konnten.

Risiken stellen nicht getroffene Architekturentscheidungen oder Eigenschaften der Architektur, die nicht vollständig
verstanden wurden, dar. So kann z.~B. der verwendete Datenbanktyp noch ungeklärt oder die Auswirkungen einer zentral geführten Komponente unklar sein.

Sensitivitätspunkte sind starke Abhängigkeiten messbarer Qualitätsmerkmale. Diese beziehen sich auf die Auslegung von einzelnen Komponenten der Architektur. Ein Beispiel für Sensitivitätspunkte ist ein Engpass zwischen zwei Modulen. D.~h. wenn der Kommunikationskanal zweier Module, wovon mindestens eins essenziell für das Gesamtsystem ist, zu gering ausgelegt wurde, dann kann dieser für einen niedrigen Durchsatz des gesamten Systems verantwortlich sein. In anderen Worten beeinflusst die Dimensionierung einer einzelnen Komponente direkt das Gesamtsystem, was einen Sensitivitätspunkt darstellt.

Zusätzlich deckt der Kompromisspunkt Sensitivitätspunkte auf, die sich entgegen wirken.
Würde man z.~B. den zuvor genannten Engpass beheben und den Kanal breiter auslegen, kann unter Umständen dies die Zuverlässigkeit des Gesamtsystems verschlechtern. Ein angebundenes Modul verwirft möglicherweise bei einer starken Belastung einige Anfragen, wodurch ein korrektes Ergebnis nicht mehr garantiert werden kann. So wirken sich diese beiden Sensitivitätspunkte entgegen und es muss ein Kompromiss gebildet werden.  
 
Mit den Risiken, Sensitivitäts- und Kompromisspunkten kann die Architektur mit gezielteren Analysen, Erstellung von Prototypen und weiteren Entwürfen stark verbessert 
werden. 

Die Grundvoraussetzung zum Erreichen dieses Zieles ist eine detaillierte Definition der Qualitätsanforderungen und eine genaue Spezifikation der Architektur sowie
dessen zugrunde liegenden Entscheidungen\cite{Clements2000}. In der Praxis ist es leider nicht unüblich, dass Ziele und Architekturdetails noch unklar oder mehrdeutig sind. Aus diesem Grund ist ein wichtiges Ziel von \ac{ATAM} auch, dies genauer zu definieren und eindeutig festzuhalten. Um die Erreichung sämtlicher Ziele zu unterstützen, ist \ac{ATAM} in 
mehrere Phasen aufgeteilt. Bild \ref{ATAMPhasen} gibt hierzu einen Überblick.

\image[ATAM-Phasen.pdf]{ATAMPhasen}{Phasen von ATAM}{Phasen der Architekturbewertung nach ATAM\cite{Starke2015}}

\subsubsection{Vorbereitung}
Das Fundament der Bewertung stellen die Qualitätsanforderungen dar. Diese können nur lückenlos bestimmt werden, wenn alle maßgeblich vom Projekt betroffenen Personen involviert sind. In der Vorbereitungs-Phase müssen somit alle wichtigen Stakeholder identifiziert werden. Neben dem Kunden bzw. Auftraggeber selbst, kann dies z.~B. der Benutzer, Administrator oder Tester sein. Dabei müssen jene Personen nicht direkt in den Prozess miteinbezogen werden. Es reicht schon einen Vertreter, meist aus dem Management, zu bestimmen oder die Wünsche und Ziele vorher genau zu ermitteln\cite{Starke2015}. In der Regel werden nur wenige Stakeholder direkt zur Architekturbewertung eingeladen. Zu viele Personen würden den Prozess, durch Diskussionen und Abschweifungen, nur verlangsamen und ineffektiv gestalten.
\subsubsection{Kickoff-Phase}\
\newparagraph{Bewertungsmethode (ATAM) vorstellen}
\
Im ersten Schritt soll die Bewertungsmethode vorgestellt und deren Ziele verdeutlicht werden. Nicht jeder Stakeholder ist regelmäßig in eine Architekturbewertung involviert. So muss klar gestellt werden, welche Bedeutung das Architektur- und Qualitätsziel hat\cite{Starke2015}. Des Weiteren sollte hervorgehoben werden, dass es bei der qualitativen Architekturbewertung um das Aufdecken von Risiken, sowie um mögliche Maßnahmen geht. Es sollte nicht Ziel sein, Noten für die Architektur zu vergeben. Insbesondere sollte die Präsentation folgende Punkte enthalten\cite{Clements2000}:

\begin{itemize}[]
	\item Eine Kurze Beschreibung von ATAM und dessen Ablauf
	\item Methoden zur Analyse sollten erklärt werden
	\item Das Ziel und die Ergebnisse der Evaluation
\end{itemize}

\newparagraph{Qualitätsziel vorstellen}
\
Die Qualitätsanforderungen werden wesentlich vom Auftraggeber bestimmt. Aus diesem Grund sollte er auch diese vorstellen und dabei genau erläutern, was die Gründe für die Entwicklung sind und wie das System in die fachlichen Unternehmensprozesse eingeordnet werden soll. Selbst wenn die Ziele bereits in einem Anforderungsdokument erfasst wurden, ist es wichtig, die aktuelle Sichtweise des Auftraggebers zu begreifen. Zudem sind Zielformulierungen aus den Anforderungsdokumenten, meist von Systemanalytikern, gründlich gefiltert worden\cite{Starke2015}. So können Teilnehmer Rückfragen stellen und \enquote{Aha}-Erlebnisse auslösen.

\newparagraph{Architektur erläutern}
\
In dieser Phase wird vom Softwarearchitekten die Architektur vorgestellt. Dabei wird nicht nur das eigentliche System erläutert, sondern es sollte der gesamte Kontext mit einbezogen werden\cite{Starke2015}. Es beinhaltet Nachbarsysteme oder Plattformen mit denen das System in Verbindung steht. Eine angemessene Detailtiefe spielt dabei auch eine Rolle. Was angemessen ist, hängt in erster Linie von den vorhandenen Informationen (Dokumente, Diagramme, usw.) und der verfügbaren Zeit ab. Dies stellt einen wesentlichen Punkt in der Bewertung dar\cite{Clements2000}, da nur die bereitgestellten Informationen in die Analyse mit einbezogen werden können. Je mehr vorhanden ist, desto tiefer und detaillierter kann die Bewertung durchgeführt werden. Wichtige und benötigte Dokumente oder Entscheidungen zur Architektur sollten unbedingt vor der Evaluation erstellt worden sein.
Die Präsentation der Architektur sollte hier folgende Informationen umfassen\cites{Clements2000}{Starke2015}:

\begin{itemize}[]
	\item Bausteine der oberen Abstraktionsebene
	\item Ausgewählte Laufzeitsichten wichtiger Use-Cases
	\item Technische Einschränkungen, wie Betriebssystem, Hardware oder Middleware
	\item Weitere Systeme mit denen dieses zusammenhängt 
\end{itemize}
\subsubsection{Bewertungsphase}\
\newparagraph{Architekturansätze identifizieren}
\
Aus der vorangegangenen Information können nun Architekturentscheidungen identifiziert werden, die zur Erfüllung spezieller Qualitätsanforderungen dienen. Es ist zu klären, wie die Architektur, strukturell oder konzeptionell, die wesentlichen Probleme oder Herausforderungen löst. Die treibenden und prägenden Architekturansätze werden dabei von den Architekten hervorgehoben und dienen der Ergänzung des vorangegangenen Überblicks\cite{Starke2015}.
So wird sichergestellt, dass die kritischen Anforderungen der Architektur erfüllt werden.

\newparagraph{Bildung eines Qualitätsbaumes (Quality Utility Tree)}
\
In dieser Phase werden alle Stakeholder aktiv. Sie identifizieren und verfeinern die wichtigsten Qualitätsanforderungen des Systems. Dabei ist es wichtig, den Fokus nicht auf die wesentlichen Anforderungen zu verlieren und sich in Feinheiten zu verstricken\cite{Clements2000}. Damit dies gelingen kann, wird ein Qualitätsbaum erstellt.
So werden die geforderten Anforderungen, z.~B. in einem Brainstorming, erfasst und anschließend in Baum-Form angeordnet. Die globalen Anforderungen stehen auf der linken Seite und werden nach rechts hin verfeinert. Bild \ref{Qualitaetsbaum} stellt ein Beispiel für einen solchen Qualitätsbaum dar. 

\image[BeispielQualitaetsbaum.pdf]{Qualitaetsbaum}{Beispiel Qualitätsbaumes}{Beispiel eines Qualitätsbaums.}

Anschließend werden zu den Qualitätsanforderungen Szenarien erstellt. Diese bilden einen wichtigen Bestandteil der Bewertungsmethode. Sie dienen dazu, die Anforderungen genauer zu definieren und den Stakeholdern somit näher zu bringen. Dabei muss die Formulierung eines Szenarios möglichst konkret sein und ein Beispiel darstellen, wie sich das System in einer bestimmten Situation verhält. Ein Szenario sollte folgende Informationen umfassen\cite{Starke2015}:

\begin{description}[leftmargin=!,labelwidth=\widthof{\bfseries Systembestandteil}]
	\item[Auslöser] Welcher spezifische Stimulus löst das Szenario aus.
	\item[Quelle] Woher stammt der Auslöser. Z. B. intern, extern, Benutzer, Administrator usw.
	\item[Umgebung] Welcher Bestandteil des Systems ist betroffen.
	\item[Antwort] Wie reagiert das System auf den Auslöser.
	\item[Metrik] Wie kann die Antwort gemessen oder bewertet werden.
\end{description}

Die Szenarien sind nicht nur eine genauere Definition der Anforderungen. Auch helfen sie den Projektbeteiligten dabei, die Qualitätseigenschaften genauer zu verstehen, da sie weniger abstrakt sind und konkrete Anwendungsfälle enthalten. Ein, um Szenarien erweiterter Qualitätsbaum, ist in Bild \ref{QualitaetsbaumSzenarien} dargestellt.

\image[BeispielQualitaetsbaumMitSzenarien.pdf]{QualitaetsbaumSzenarien}{Qualitätsbaum und Szenarien}{Qualitätsbaum mit einzelnen Szenarien erweitert}

Zuletzt werden den Szenarien Prioritäten zugeordnet, so dass der Fokus zuerst auf kritischen Anforderungen liegt. Auch ist der Bewertungs-Workshop meist zeitlich begrenzt und somit wird sichergestellt, dass die Nichterfüllung essentieller Qualitätsanforderungen schnell zu einem Abbruch führt. Es kann durch eine einfache Skala mit wenigen Prioritäten erreicht werden, z.~B. A = hoch, B = mittel und C = weniger wichtig.

\newparagraph{Architekturansätze analysieren}
\
Anhand der Prioritäten kann nun die eigentliche Analyse beginnen. In kleineren Gruppen, zusammen mit den Architekten oder Entwicklern, werden die Szenarien genauer untersucht und zugehörige Architekturansätze erläutert.
So kann z.~B. mittels eines Walkthrough genau aufgezeigt werden, wie einzelne Komponenten zur Erreichung des Ziels interagieren und welche Entwurfsentscheidungen unterstützen\cite{Starke2015}. Dabei sollten folgende Fragen geklärt werden: 

\begin{itemize}
	\item Wie wird das Szenario in der Architektur umgesetzt?
	\item Warum wurde dieser Architekturansatz gewählt?
	\item Gibt es Kompromisse, die gemacht wurden?
	\item Werden andere Qualitätsmerkmale davon beeinflusst?
	\item Gibt es Risiken, die damit verbunden sind?
	\item Wird die Entscheidung von Analysen, Untersuchungen oder Prototypen unterstützt?
\end{itemize}

Die Analyse ist dabei nicht auf diese Fragen beschränkt. Sie bieten lediglich einen Startpunkt für eine Diskussion, um potentielle Risiken, Sensitivitäts- oder Kompromisspunkte zu finden. An dieser Stelle gibt es leider kein Patentrezept oder ein spezifisches Vorgehen, um das Erfüllen der Qualitätsanforderung zu bestimmen. Viel mehr wird von den Teilnehmern ein analytisches und systematisches Denken verlangt. Im Fokus muss hierbei stehen, dass eine Verbindung zwischen der Architekturentscheidung und der Anforderung geschaffen wird.

\subsubsection{Abschlussphase}
Am Ende wird das Ergebnis den Stakeholdern präsentiert. Es muss nicht zwangsläufig nur durch eine Präsentation dargestellt, sondern kann auch um einen detaillierten Bericht ergänzt werden. Sämtliche Phasen der Bewertung und dessen Ergebnisse können so noch einmal rekapituliert werden und bilden die Basis zur Definition von eventuell benötigten Maßnahmen.

Die durch \ac{ATAM} gefundenen Risiken und die daraus resultierenden Maßnahmen können auch Änderungen an der Architektur darstellen. Daher bietet sich \ac{ATAM} auch zum iterativen Einsatz an. Mit ersten Entwürfen einer komplexen Architektur kann überprüft werden, ob die Entscheidungen den richtigen Weg einschlagen und wichtige Anforderungen von Beginn an mit einbezogen werden. So bildet sich Schrittweise eine detaillierte Architektur unter gut dokumentierter Berücksichtigung der Qualitätsanforderungen. 

\subsection{\acf*{SAEM}}
Einen anderen Ansatz verfolgt die \ac{SAEM} Methode. Diese betrachtet die Softwarearchitektur als finales Produkt der Entwurfsphase und versucht mittels dem ISO/IEC 25010\footnote{
	In der ursprünglichen Veröffentlichung von \ac{SAEM} wurde das ISO/IEC 9126-1 Modell verwendet. Die ISO/IEC 9126-1 wurde 2011 durch die ISO/IEC 25010 ersetzt.
}
Qualitätsmodell die Qualität der Architektur festzustellen. Nicht nur weil Fehler in der Architektur schwer zu beheben sind, sondern diese auch direkt einen Einfluss auf die Qualität und Kapazitäten des Systems haben\cite{Duenas1998}. Dies trifft besonders auf die nicht-funktionalen Anforderungen wie Wartbarkeit, Skalierbarkeit oder Effizienz zu.\\
Im Gegensatz zu \ac{ATAM} ist \ac{SAEM} jedoch weniger bekannt und erprobt\cite{IEEE_TSE2002}

Wie beim finalen System ist die direkte Messung am Produkt eine objektive und sehr effektive Evaluation der Qualität. Um dies zu erreichen, müssen Qualitätsanforderungen definiert und ein Prozess zur Überprüfung geplant, implementiert und kontrolliert werden\cite{Duenas1998}. \ac{SAEM} versucht das mittels eines Qualitätsmodells auf die Softwarearchitektur abzubilden und stellt Methoden sowie Metriken für die Evaluation bereit. 

\subsubsection{Spezifikation Qualität}
Das Qualitätsmodell von \ac{SAEM} baut dabei auf dem Modell der ISO/IEC 25010 auf. Es teilt die Qualität in 8 Kategorien und weitere Unterkategorien auf\cite{ISOIEC25010}. Dadurch kann die Qualität der Softwarearchitektur spezifiziert werden, indem man diesen Architekturanforderungen zuordnet.

\image[ISO25010.pdf]{ISO25010Modell}{Qualitätsmodell der ISO/IEC 25010}{Qualitätsmodell der ISO/IEC 25010\cite{ISOIEC25010}}

Hierzu wird im ersten Schritt die externe Qualität bestimmt. Diese basiert auf messbaren Metriken unter Nutzungsbedingungen und werden vom Nutzer sowie den Architekten definiert\cite{Duenas1998}. So sollten für alle Qualitätskategorien Anforderungen an die Architektur ausgewählt und optimale bzw. zulässige Werte bestimmt werden. Ein Beispiel hierfür ist die Reaktionszeit auf Anfragen.\\ 
Mittels der externen Qualität können die Architekten die interne Qualität bestimmen. Diese sind Anforderungen, die benötigt werden um das Ziel zufriedenstellend zu erreichen. Im Unterschied zu externen Anforderungen werden diese nicht durch die Nutzung bestimmt, sondern beziehen sich auf intrinsische Eigenschaften, wie z.~B. Modularität oder Komplexität\cite{Duenas1998}. Die Bestimmung der internen Qualität basiert meist auf Expertenwissen oder firmeninterne Richtlinien.

\subsubsection{Bestimmung von Metriken}
Für die Messung der Qualitätsanforderungen müssen Metriken bestimmt werden. Hierzu kann das Bewertungsteam aus bekannten Softwaremetriken wählen oder systematisch neue definieren. Wichtig ist nur, dass die Messung zielorientiert ist. D. h. die Metriken müssen mit einem Top-Down Ansatz bestimmt werden\cite{Habenicht2008}. Somit soll nicht aus vorhandenen Messungen ein Bezug zur Qualität aufgebaut, sondern mittels spezifischer Qualitätsanforderungen passende und aussagekräftige Metriken gefunden werden.

Eine sehr gute Vorlage hierzu ist die \ac{GQM} Methode. Diese setzt die Qualitätsanforderungen, die das Ziel darstellen, an erste Stelle. Auf dessen Basis werden Fragen abgeleitet, die es wiedergeben. Dabei muss es pro Ziel nicht nur eine Frage geben. In vielen Fällen ist eine Komposition aus verschiedensten Fragen notwendig. Dabei hilft es, das Ziel aus mehreren Sichten zu betrachten. Nach\cite{Habenicht2008}
sind Teilkriterien zur Erfüllung der notwendigen Fragestellung:
 
\begin{description}[leftmargin=!,labelwidth=\widthof{\bfseries Anwendungsbereich}]
	\item[Sichtweise] Auftraggeber, Administrator, Entwickler usw.
	\item[Anwendungsbereich] Benötigte Technologien, angeschlossene Systeme, usw.
	\item[Zweck] Analyse, Kontrolle, Verständnis, usw.
	\item[Kontext] intern, extern
\end{description} 
 
Anschließend werden für sämtliche Fragestellungen Metriken bestimmt, die diese am besten beantworten. Die damit erhobenen Daten können dabei objektiv (z.~B. Unterstützung Standardformate) oder subjektiv (z.~B. Skala für Zufriedenstellung) sein.
Durch den Top-Down Ansatz stehen die erhobenen Daten immer im Zusammenhang mit den Zielen, womit die Interpretation leichter fällt. 

Ein Beispiel für das Ergebnis von \ac{GQM} wird in Bild \ref{GQMBeispiel} dargestellt.

\image[GQMBeispiel.pdf]{GQMBeispiel}{Beispiele \acs{GQM}}{Beispiel für die Anwendung von \ac{GQM} an einer Microservice Architektur}
 
\subsubsection{Evaluation}
An diesem Punkt ist die Evaluation denkbar einfach. Sie besteht aus dem Sammeln und Auswerten von Daten mittels Metriken. Über die gewonnenen Daten und die vorher bestimmten zulässigen Werte, wie das Vorhandensein spezifischer Funktionen, wird die Architekturqualität bestimmt. Es lässt sich auch eine Aussage über die zu erwartende Qualität des Endsystems treffen.
An diesem Punkt wird der Unterschied zu \ac{ATAM} wieder deutlich. \ac{SAEM} versucht nicht ein tieferes Verständnis für die Architektur bei maßgeblichen Stakeholdern zu schaffen, sondern nur die Einhaltung spezifischer Anforderungen zu gewährleisten. Dies hängt dabei stark vom definiertem Qualitätsziel ab. Gerade bei der internen Qualität verlangt \ac{SAEM} somit eine firmeninterne Qualitätsrichtlinie\cite{IEEE_TSE2002}, die auf Erfahrung und Expertenwissen basiert.

\subsection{Architekturbewertungsmethoden und Frameworks}
Mit Architekturbewertungsmethoden wie \ac{ATAM} oder \ac{SAEM} lässt sich die Qualität einer Architektur sehr gut bewerten. So kann die spezifische Umsetzung einer Microservice Architektur auf die gegebenen Anforderungen untersucht werden. Zudem schärft es nicht nur die Anforderungen selbst, sondern kann auch die geplante Architektur weiter verfeinern und ein tieferes Verständnis bei den Stakeholdern dafür schaffen.

Mit diesem Wissen müssen anschließend die passenden Frameworks für die einzelnen Services gewählt werden. Hierzu lässt sich die reine Architekturbewertung jedoch nicht mehr verwenden, da diese ein vollständiges Verständnis über die Architektur und dessen Einfluss benötigt. D.~h., in Bezug auf ein Framework muss bekannt sein, wie sich die vorhandenen Architekturentscheidungen und Entwurfsmuster innerhalb des Frameworks auf die tatsächliche Architektur der Umsetzung eines Microservices auswirken.\\
Beim Einsatz eines neuen Frameworks ist dieses Wissen jedoch in vielen Fällen nicht vorhanden und muss erarbeitet werden. Aus diesem Grund ist es verständlich, wenn Entwickler und Architekten bei der Wahl eines Frameworks auf etwas zurück greifen, dass bekannt ist. Eine Bewertungsmethode sollte somit kein Expertenwissen voraussetzen, damit sich alle Frameworks in den Bewertungsprozess mit einbeziehen lassen.




