\section{Fazit und Ausblick}

Mit \ac{MFEM} wurde eine Methode vorgestellt, die ein Framework von allen Seiten zielgerichtet auf eine Microservice Architektur hin untersucht. Sie bietet viele Ansätze, um den Prozess so effizient wie möglich zu gestalten. Dabei verliert sie nicht an Flexibilität und kann an die jeweilige Situation angepasst werden. Dies geht dabei über die Wahl der passenden Anforderungen und Szenarien hinaus.

\subsection{\ac*{MFEM} Anpassung/Erweiterung}

Der wohl größte Kritikpunkt an \ac{MFEM} ist der hohe Zeitaufwand für die Evaluation. Auch wenn für die voll umfängliche Bewertung eines Frameworks, ohne dabei Vorkenntnisse zu verlangen, Zeit benötigt wird, kann diese verkürzt werden. Anstatt sämtliche Anforderungen in die Evaluationsphase zu überführen, kann anhand der Prioritäten eine Auswahl getroffen werden.\\
So können z.~B. nur Anforderungen mit der Priorität \enquote{A} untersucht werden. Dies hat den Vorteil, dass sich die Evaluation erheblich verkürzt. Im Gegenzug ergeben sich so jedoch blinde Flecken, die ein größeres Restrisiko hinterlassen.
 
Des Weiteren könnte \ac{MFEM} auch durch eine Langzeitsicht erweitert werden. Eine regelmäßige Bewertung, z.~B. pro Major Release, würde einen Trend aufzeigen. So werden vielleicht in späteren Versionen benötigte Features nicht mehr unterstützt oder ein anfänglich schlechtes Framework entwickelt sich zur positiven Alternative.

\subsection{Ausblick}

In jedem Fall wäre eine Anwendung von \ac{MFEM} auf die aktuell relevanten Frameworks interessant. Anhand der allgemeinen Basisanforderungen ließe sich sogar ein Ranking aufstellen. Viele derzeit publizierte Rankings, wie \cite{JaxFrameTrend2017} und \cite{HotFramework2017}, betrachten nur einen Aspekt oder beziehen sich rein auf Umfragen.\\ 
Mit \ac{MFEM} würde hingegen eine detailliertere Bewertung erfolgen, aufgrund dessen fundierte Technologieeinscheidungen getroffen werden können.