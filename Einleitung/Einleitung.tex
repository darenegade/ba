\section{Einleitung}

Mit der Microservice Architektur lassen sich verschiedenste Technologien vermischen. Jeder Service kann mit unterschiedlichen Frameworks und Programmiersprachen entwickelt werden.\\
Diese Freiheit spiegelt sich im Auswahlprozess der Softwareentwickler wider. Sie laufen unter Umständen einem Hype hinterher\cite{HDD2016} und die Erwartungen an das neue Framework sind häufig stark überzogen. Dieser Investition in die neue Technologie folgt Ernüchterung. Im Detail sind viele Funktionen oft nicht so, wie man es erwartet hat und im schlimmsten Fall schlägt die Entwicklung fehl.

Zudem ist die Microservice Architektur nicht eindeutig definiert\cite[11]{Wolff2015}. Jede Ausprägung ist anderes und folgt den spezifischen Anforderungen oder dem Geschmack des Architekten. So kann ein Framework für eine Ausprägung ideal und für eine andere ein regelrechter Fehlgriff sein. 

Dies bietet jedoch keinen Anlass sämtliche unbekannten Frameworks zu meiden und nur die jahrelang Etablierten zu nutzen. Neue Technologien können auch eine Chance bieten und echten Mehrwert produzieren. Sie lösen mitunter ein gravierendes Problem oder die Entwicklungszeit verringert sich immens. So lohnt sich gerade in einer Microservice Architektur das Risiko einzugehen, da der Fehlschlag eines einzelnen Services vertretbar ist.

Um dieses Risiko weiter zu minimieren und das richtige Framework zu finden, soll in dieser Arbeit eine Bewertungsmethode für Microservice Frameworks etabliert werden. Damit verbunden, sollen folgende Eigenschaften von der Methode erfüllt werden:

\begin{description}
	\item[Sprachunabhängigkeit] 
	So wie die Erstellung eines Microservices in jeder Programmiersprache erfolgen kann, soll auch die Methode unabhängig sein.
	\item[Kein Vorwissen nötig] 
	Damit sämtliche Frameworks einbezogen werden können, darf die Methode vom Bewertungsteam kein Vorwissen über das Framework verlangen.
	\item[Anpassbarkeit] 
	Die Methode muss in Abhängigkeit zu einer spezifischen Ausprägung der Architektur angepasst werden können.
	\item[Vergleichbarkeit] 
	Das Ergebnis der Methode soll sowohl für sich stehen, als auch ein Vergleich zu anderen Frameworks eröffnen.
\end{description}

Mit der Bewertung lässt sich eine Microservice Architektur aufbauen, die technologisch Heterogen ist und bei der Chancen genutzt sowie Risiken minimiert werden.