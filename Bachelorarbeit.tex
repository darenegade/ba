\documentclass[12pt,a4paper,bibliography=totocnumbered,listof=totocnumbered,final]{scrartcl}
\usepackage[T1]{fontenc}
\usepackage[ngerman]{babel}
\usepackage[utf8]{inputenc}
\usepackage{amsmath}
\usepackage{amsfonts}
\usepackage{amssymb}
\usepackage{graphicx}
\usepackage{fancyhdr}
\usepackage{tabularx}
\usepackage{geometry}
\usepackage{setspace}
\usepackage[right]{eurosym}
\usepackage[printonlyused]{acronym}
\usepackage{subfig}
\usepackage{floatflt}
\usepackage[usenames,dvipsnames]{color}
\usepackage{colortbl}
\usepackage{paralist}
\usepackage{array}
\usepackage{titlesec}
\usepackage{parskip}
\usepackage[right]{eurosym}
\usepackage[subfigure,titles]{tocloft}
\usepackage[pdfpagelabels=true]{hyperref}
\usepackage[rgb]{xcolor}
\usepackage[obeyFinal, colorinlistoftodos, backgroundcolor=yellow, 
linecolor=yellow, textsize=tiny, textwidth=43]{todonotes}
\usepackage{letltxmacro}
\usepackage[autostyle, german=quotes]{csquotes}
\MakeOuterQuote{"}
\usepackage{biblatex}
\usepackage{enumitem}
\usepackage{etoolbox}
\usepackage{xparse}
\usepackage{import}
\usepackage[pagewise, right]{lineno}
\usepackage{ifdraft}
\usepackage{multirow}
\usepackage{placeins}
\usepackage{longtable}
\usepackage[toc,numberedsection,nopostdot]{glossaries}
\usepackage{listings}

\geometry{a4paper, top=27mm, left=30mm, right=20mm, bottom=35mm, headsep=10mm, footskip=12mm}

\hypersetup{unicode=false, pdftoolbar=true, pdfmenubar=true, pdffitwindow=false, pdfstartview={FitH},
	pdftitle={Bachelorarbeit},
	pdfauthor={Rene Zarwel},
	pdfsubject={Microservices und technologische Heterogenität},
	pdfcreator={\LaTeX\ with package \flqq hyperref\frqq},
	pdfproducer={pdfTeX \the\pdftexversion.\pdftexrevision},
	pdfkeywords={Bachelorarbeit},
	pdfnewwindow=true,
	colorlinks=true,linkcolor=black,citecolor=black,filecolor=magenta,urlcolor=black}
\pdfinfo{/CreationDate (D:20110620133321)}

%-------Listings Definitions------
\definecolor{red}{rgb}{0.6,0,0} % for strings
\definecolor{green}{rgb}{0.25,0.5,0.35} % comments
\definecolor{purple}{rgb}{0.5,0,0.35} % keywords
\definecolor{docblue}{rgb}{0.25,0.35,0.75} % javadoc
\lstset{
	basicstyle=\footnotesize,
	keywordstyle=\color{purple}\bfseries,
	stringstyle=\color{red},
	commentstyle=\color{green},
	morecomment=[s][\color{docblue}]{/**}{*/},
	numbers=left,
	numberstyle=\tiny\color{black},
	stepnumber=2,
	tabsize=4,
	showspaces=false,
	showstringspaces=false
	captionpos=b, 
	breaklines=true, 
	frame=lines, 
	numberstyle=\tiny, 
	xleftmargin=2em, 
	framexleftmargin=2em
}
\makeatletter
\def\l@lstlisting#1#2{\@dottedtocline{1}{0em}{1em}{\hspace{1,5em} Lst. #1}{#2}}
\makeatother
\lstdefinelanguage{Golang}
{morekeywords=[1]{package,import,func,type,struct,return,defer,panic,%
		recover,select,var,const,iota,},%
	morekeywords=[2]{string,uint,uint8,uint16,uint32,uint64,int,int8,int16,%
		int32,int64,bool,float32,float64,complex64,complex128,byte,rune,uintptr,%
		error,interface},%
	morekeywords=[3]{map,slice,make,new,nil,len,cap,copy,close,true,false,%
		delete,append,real,imag,complex,chan,},%
	morekeywords=[4]{for,break,continue,range,goto,switch,case,fallthrough,if,%
		else,default,},%
	morekeywords=[5]{Println,Printf,Error,},%
	sensitive=true,%
	morecomment=[l]{//},%
	morecomment=[s]{/*}{*/},%
	morestring=[b]',%
	morestring=[b]",%
	morestring=[s]{`}{`},%
}
%------------------------------

\numberwithin{figure}{section} %Numbering of figs include section
\numberwithin{table}{section} %Numbering of tables include section

\setlist[description]{labelindent=1cm}
\setlist[itemize]{labelindent=1cm}
\setlist[enumerate]{labelindent=1cm}

\newcommand{\newparagraph}[1]{\paragraph{#1}~\\~\\}
\newcommand{\newsubparagraph}[1]{\subparagraph{#1}~\\}

\newcommand{\acfootnote}[1]{\acs{#1}\footnote{\acf{#1}}}

% Nice and big quote
% #1 Quote
% #2 Name of person to quote
\newcommand{\bigquote}[2]{
	\vspace*{1\baselineskip}
	\begin{center}
		\begin{minipage}{0.8\textwidth}
			\enquote{\textit{#1}}\par
			- #2
		\end{minipage}
	\end{center}
	\vspace*{1\baselineskip}
}
%-----------------

%Todo Commands
\newcounter{todocounter}

\LetLtxMacro{\todoold}{\todo}
\renewcommand{\todo}[2][]{\stepcounter{todocounter}\todoold[#1]{TODO \thetodocounter: #2}}
\newcommand{\todoInline}[1]{\todo[inline]{#1}}
\newcommand{\todoFigure}[1]{\stepcounter{todocounter}\missingfigure[figcolor=white]{TODO \thetodocounter: #1}}
% ----------------

%Cite Redefinition - Auto Space on citation + Prevent forever alone citation
\renewcommand{\multicitedelim}{\addsemicolon\space}
\LetLtxMacro{\citeold}{\cite}
\renewcommand{\cite}[2][]{~\citeold[#1]{#2}}
\LetLtxMacro{\citesold}{\cites}
\DeclareDocumentCommand{\cites}{o m o m}{
	\IfNoValueTF{#1}{
		\IfNoValueTF{#3}{
			~\citesold{#2}{#4}
		}{
			~\citesold{#2}[#3]{#4}
		}
	}{
		\IfNoValueTF{#3}{
			~\citesold[#1]{#2}{#4}
		}{
			~\citesold[#1]{#2}[#3]{#4}
		}
	}
}
%----------------------

%Image Command - Optional Image otherwise TODO
% #1 IMAGE File - optional (TODO if not present)
% #2 Image options - optional
% #3 LabelName
% #4 TOC Text
% #5 Captiontext
\DeclareDocumentCommand{\image}{ o O{width=\linewidth} m m m}{
	\begin{minipage}{\linewidth}
		\vspace*{0.5cm}
		\centering
		\IfNoValueTF{#1}{
			\todoFigure{#4}
		}{
			\includegraphics[#2]{Bilder/#1}
		}	
		\captionof{figure}[#4]{#5}
		\label{#3}
		\vspace*{0.5cm}
\end{minipage}
}
%-------------

\makeglossaries
\loadglsentries{glossar}
\glsaddall
\bibliography{bibo}

\begin{document}

\titlespacing{\section}{0pt}{12pt plus 4pt minus 2pt}{-6pt plus 2pt minus 2pt}

% Kopf- und Fusszeile
\renewcommand{\sectionmark}[1]{\markright{#1}}
\renewcommand{\leftmark}{\rightmark}
\pagestyle{fancy}
\lhead{}
\chead{}
\rhead{\thesection\space\contentsname}
\lfoot{Microservices und technologische Heterogenität}
\cfoot{}
\rfoot{\ \linebreak Seite \thepage}
\renewcommand{\headrulewidth}{0.4pt}
\renewcommand{\footrulewidth}{0.4pt}

% Vorspann
\renewcommand{\thesection}{\Roman{section}}
\renewcommand{\theHsection}{\Roman{section}}
\pagenumbering{Roman}

% ----------------------------------------------------------------------------------------------------------
% Titelseite
% ----------------------------------------------------------------------------------------------------------
\thispagestyle{empty}
\begin{center}
	\includegraphics[scale=0.5]{Bilder/Hochschule_Muenchen_Logo.png}\\
	\vspace*{2cm}
	\Large
	\textbf{Fakultät für Informatik und Mathematik 07}\\
	\vspace*{2cm}
	\Huge
	\textbf{Bachelorarbeit}\\
	\vspace*{0.5cm}
	\large
	über das Thema\\
	\vspace*{1cm}
	\textbf{Microservices und technologische Heterogenität}\\
	Entwicklung einer sprachunabhängigen Microservice Framework Evaluationsmethode\\
	\vspace*{2cm}
	
	\vfill
	\normalsize
	\newcolumntype{x}[1]{>{\raggedleft\arraybackslash\hspace{0pt}}p{#1}}
	\begin{tabular}{x{6cm}p{7.5cm}}
		\rule{0mm}{5ex}\textbf{Autor:} & René Zarwel\newline zarwel@hm.edu \\ 
		\rule{0mm}{5ex}\textbf{Prüfer:} & Prof. Dr. Hammerschall \\ 
		\rule{0mm}{5ex}\textbf{Abgabedatum:} & 13.03.2017\\ 
	\end{tabular} 
\end{center}
\pagebreak

% ----------------------------------------------------------------------------------------------------------
% Abstract
% ----------------------------------------------------------------------------------------------------------
\setcounter{page}{1}
\onehalfspacing
\titlespacing{\section}{0pt}{12pt plus 4pt minus 2pt}{2pt plus 2pt minus 2pt}
\rhead{KURZFASSUNG}
\section{Kurzfassung}
Durch den Erfolg der Microservices entstehen immer mehr zugehörige Frameworks. So wird in dieser Arbeit eine Bewertungsmethode entwickelt, um die Microservice Frameworks anhand einer gegebenen Architektur zu bewerten und miteinander zu vergleichen. Sie wird dabei stark an die Bewertung von Softwarearchitekturen angelehnt, da ein Framework einen erheblichen Einfluss darauf hat. Aus diesem Grund werden diese genauer betrachtet, bevor die neue Methode eingeführt wird.\\
Die so vorgestellte \acf*{MFEM} wird anschließend an zwei Beispielen evaluiert, um die Wirksamkeit der Methode zu überprüfen. So hat sich gezeigt, dass die entwickelte Methode sehr gute Ergebnisse liefert und eine fundierte Entscheidungsgrundlage ist.  

\vspace{-1,2em}
\titlespacing{\section}{0pt}{12pt plus 4pt minus 2pt}{-6pt plus 2pt minus 2pt}
\section*{Abstract}
Due to the success of Microservices, more and more related frameworks are emerging. In this thesis a method of evaluation is developed in order to evaluate and compare the microservice frameworks on the basis of a given architecture. The method is heavily based on the evaluation of software architectures since a framework has a considerable influence on them. For this reason, they are considered more closely before the new method is introduced.\\
Finally the presented \acf*{MFEM} is evaluated with two examples to check the effectiveness of the method. This evaluation has yielded very good results and \acs*{MFEM} is a well-founded decision making.
\pagebreak

% ----------------------------------------------------------------------------------------------------------
% Verzeichnisse
% ----------------------------------------------------------------------------------------------------------
\renewcommand{\cfttabpresnum}{Tab. }
\renewcommand{\cftfigpresnum}{Abb. }
\settowidth{\cfttabnumwidth}{Abb. 10\quad}
\settowidth{\cftfignumwidth}{Abb. 10\quad}
\setlength{\cftsecnumwidth}{2em}
\setlength{\cftsubsecindent}{1cm}
\setlength{\cftsubsubsecindent}{2cm}

\titlespacing{\section}{0pt}{12pt plus 4pt minus 2pt}{2pt plus 2pt minus 2pt}
\singlespacing
\rhead{INHALTSVERZEICHNIS}
\renewcommand{\contentsname}{II Inhaltsverzeichnis}
\phantomsection
\addcontentsline{toc}{section}{\texorpdfstring{II \hspace{0.35em}Inhaltsverzeichnis}{Inhaltsverzeichnis}}
\addtocounter{section}{1}
\tableofcontents
\pagebreak
\rhead{VERZEICHNISSE}
\listoffigures
\pagebreak
\listoftables
%\pagebreak
\renewcommand{\lstlistlistingname}{Listing-Verzeichnis}
{\labelsep2cm\lstlistoflistings}
\pagebreak
% ----------------------------------------------------------------------------------------------------------
% Abkürzungen
% ----------------------------------------------------------------------------------------------------------
\section{Abkürzungsverzeichnis}
\begin{acronym}[HATEOAS] % längste Abkürzung steht in eckigen Klammern
	\setlength{\itemsep}{-\parsep} % geringerer Zeilenabstand
	\acro{SEI}{Software Engineering Institute}
	\acro{SAAM}{Software Architecture Analysis Method}
	\acro{ATAM}{Architecture Tradeoff Analysis Method}
	\acro{SAEM}{Software Architecture Evaluation Model}
	\acro{GQM}{Goal Question Metrik}
	\acro{QUT}{Quality Utility Tree}
	\acro{MFEM}{Microservice Framework Evaluation Method}
	\acro{HATEOAS}{Hypermedia As The Engine Of Application State}
	\acro{REST}{Representational State Transfer}
	\acro{XML}{Extensible Markup Language}
	\acro{JSON}{JavaScript Object Notation}
	\acro{CW}{Cognitive Walkthrough}
	\acro{JWT}{JSON Web Token}
	\acro{API}{Programmierschnittstelle}
	\acro{JPA}{Java Persistence API}
	\acro{LOC}{Lines of Code}
\end{acronym}
\newpage
% ----------------------------------------------------------------------------------------------------------
% Glossar
% ----------------------------------------------------------------------------------------------------------
\printglossary[style=altlist,nonumberlist]
\newpage
% ----------------------------------------------------------------------------------------------------------
% Inhalt
% ----------------------------------------------------------------------------------------------------------
% Abstände Überschrift
\titlespacing{\section}{0pt}{12pt plus 4pt minus 2pt}{-6pt plus 2pt minus 2pt}
\titlespacing{\subsection}{0pt}{12pt plus 4pt minus 2pt}{-6pt plus 2pt minus 2pt}
\titlespacing{\subsubsection}{0pt}{12pt plus 4pt minus 2pt}{-6pt plus 2pt minus 2pt}

% Kopfzeile
\renewcommand{\sectionmark}[1]{\markright{#1}}
\renewcommand{\subsectionmark}[1]{}
\renewcommand{\subsubsectionmark}[1]{}
\lhead{Kapitel \thesection}
\rhead{\rightmark}

\onehalfspacing
\renewcommand{\thesection}{\arabic{section}}
\renewcommand{\theHsection}{\arabic{section}}
\setcounter{section}{0}
\pagenumbering{arabic}
\setcounter{page}{1}

%
\ifoptionfinal{}{\linenumbers}
%----------------------------------------------------------------------------------------------------------
% Einleitung und Motivation
% ----------------------------------------------------------------------------------------------------------
\subimport{Einleitung/}{Einleitung}
\pagebreak
% ----------------------------------------------------------------------------------------------------------
% Microservices
% ----------------------------------------------------------------------------------------------------------
\subimport{Grundlagen/}{Grundlagen}
\pagebreak

% ----------------------------------------------------------------------------------------------------------
% Konzeption Methode
% ----------------------------------------------------------------------------------------------------------
\subimport{Konzeption/}{Konzeption}

\pagebreak

% ----------------------------------------------------------------------------------------------------------
% Evaluation an Beispielen
% ----------------------------------------------------------------------------------------------------------
\subimport{Evaluation/}{Evaluation}

\pagebreak

% ----------------------------------------------------------------------------------------------------------
% Zusammenfassung und Ausblick
% ----------------------------------------------------------------------------------------------------------
\subimport{Zusammenfassung/}{Zusammenfassung}

\pagebreak
\nolinenumbers
% ----------------------------------------------------------------------------------------------------------
% Literatur
% ----------------------------------------------------------------------------------------------------------
\renewcommand\refname{Quellenverzeichnis}

\printbibliography
\pagebreak

% ----------------------------------------------------------------------------------------------------------
% Todos
% ----------------------------------------------------------------------------------------------------------
\todototoc
\listoftodos
\pagebreak
% ----------------------------------------------------------------------------------------------------------
% Selbstständigkeitserklärung
% ----------------------------------------------------------------------------------------------------------
\begin{appendix}
\pagenumbering{Roman}
\setcounter{page}{1}

\thispagestyle{empty}
\section*{Selbstständigkeitserklärung}
\phantomsection
\addcontentsline{toc}{section}{Selbstständigkeitserklärung}
\addtocontents{toc}{\vspace{-0.5em}}

\vspace{2cm}
\begin{table}[!h]
	\centering
	\begin{tabular}{p{0.5\linewidth}p{0.5\linewidth}}
		\textbf{Zarwel, René} & \textbf{München, 13.03.2017} \\ 
		(Familienname, Vorname) & (Ort, Datum)\\
		\\ \\
		\textbf{21.10.1987} & \textbf{Informatik - 7B / WS - 2013} \\ 
		(Geburtsdatum) & (Studiengruppe / WS/SS)\\
	\end{tabular} 
\end{table}
\vspace{2cm}

\begin{center}
	\Large
	\textbf{Erklärung}
\end{center}

\vspace{1cm}
Hiermit erkläre ich, dass ich die Bachelorarbeit selbständig verfasst, noch nicht anderweitig für Prüfungszwecke vorgelegt, keine anderen als die angegebenen Quellen oder Hilfsmittel benutzt sowie wörtliche und sinngemäße Zitate als solche gekennzeichnet habe.\\[4ex]

München, den 13.03.2017\\[6ex]
\flushleft
\newlength\us
\settowidth{\us}{-René~Zarwel-}
\begin{tabular}{p{\us}}\hline
	\centering\footnotesize René~Zarwel
\end{tabular}
\pagebreak
% ----------------------------------------------------------------------------------------------------------
% Anhang
% ----------------------------------------------------------------------------------------------------------
\lhead{Anhang \thesection}
\section*{Anhang}
\phantomsection
\addcontentsline{toc}{section}{Anhang}
\addtocontents{toc}{\vspace{-0.5em}}
\section{\acs*{MFEM} - Vollständiger Quality Utility Tree der Basisanforderungen}
	\begin{minipage}{\linewidth}
	\centering
	\includegraphics[width=0.8\linewidth]{Bilder/Basisanforderungen-QUT.pdf}	
	\end{minipage}
\pagebreak

\end{appendix}
\end{document}
