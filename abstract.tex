\onehalfspacing
\titlespacing{\section}{0pt}{12pt plus 4pt minus 2pt}{2pt plus 2pt minus 2pt}
\rhead{KURZFASSUNG}
\section{Kurzfassung}
Durch den Erfolg der Microservices entstehen immer mehr zugehörige Frameworks. So wird in dieser Arbeit eine Bewertungsmethode entwickelt, um die Microservice Frameworks anhand einer gegebenen Architektur zu bewerten und miteinander zu vergleichen. Sie wird dabei stark an die Bewertung von Softwarearchitekturen angelehnt, da ein Framework einen erheblichen Einfluss darauf hat. Aus diesem Grund werden diese genauer betrachtet, bevor die neue Methode eingeführt wird.\\
Die so vorgestellte \acf*{MFEM} wird anschließend an zwei Beispielen evaluiert, um die Wirksamkeit der Methode zu überprüfen. So hat sich gezeigt, dass die entwickelte Methode sehr gute Ergebnisse liefert und eine fundierte Entscheidungsgrundlage ist.  

\vspace{-1,2em}
\titlespacing{\section}{0pt}{12pt plus 4pt minus 2pt}{-6pt plus 2pt minus 2pt}
\section*{Abstract}
Due to the success of Microservices, more and more related frameworks are emerging. In this thesis a method of evaluation is developed in order to evaluate and compare the microservice frameworks on the basis of a given architecture. The method is heavily based on the evaluation of software architectures since a framework has a considerable influence on them. For this reason, they are considered more closely before the new method is introduced.\\
Finally the presented \acf*{MFEM} is evaluated with two examples to check the effectiveness of the method. This evaluation has yielded very good results and \acs*{MFEM} is a well-founded decision making.