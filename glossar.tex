\newglossaryentry{Framework}{
	name=Framework,
	description={Ein Framework bietet einen Rahmen zur Erstellung einer Anwendung. Dabei ist es selber noch kein fertiges Programm. Es gibt vielmehr ein Grundgerüst vor und unterstützt den Entwickler soweit es geht. Dabei bietet es eine Designgrundlage und beeinflusst maßgeblich die Architektur.}
}
\newglossaryentry{Qualitaet}{
	name=Qualität,
	description={Nach der DIN EN ISO 9000:2015-11 ist Qualität, der \enquote{Grad, in dem ein Satz inhärenter Merkmale eines Objekts Anforderungen erfüllt}\cite[Kap.~3.1.1]{ISO90002015}. Sie gibt somit das Ausmaß an, wie ein Produkt den bestehenden Anforderungen nachkommt. Dabei sind \enquote{inhärente Merkmale} die Merkmale, die dem Produkt innewohnen. Sie grenzen sich damit von den zugeordneten Merkmalen ab, wie z.~B. dem Preis.}
}
\newglossaryentry{Anforderung}{
	name=Anforderung,
	description={
		Die Grundlage einer jeden Bewertung sind spezifisch definierte Anforderungen. Diese stellen die notwendige Beschaffenheit oder Fähigkeit eines Produktes dar, dass es für die Abnahme erfüllen muss. Dabei ist eine Unterscheidung von funktionalen und nicht-funktionalen Anforderungen weit verbreitet. Funktionale Anforderungen bestimmen, \textit{was} das Produkt tun soll\cite{Robertson2006}.
		Nicht-funktionalen Anforderungen beschreiben \textit{in welcher Qualität} die zu erbringenden Leistung erfolgen soll\cite{Robertson2006}.
		In der Regel werden diese Anforderungen durch Verträge, Normen oder Spezifikationen festgelegt.
	}
}
\newglossaryentry{Merkmale}{
	name=Merkmale,
	description={Merkmale sind charakteristischen Kennzeichen eines Produktes. Im Allgemeinen geben sie an, wie sich das Produkt von anderen unterscheidet. Zudem ergibt eine Messung der Merkmale am Produkt Aufschluss über die Erfüllung der Anforderungen. }
}
\newglossaryentry{Ziele}{
	name=Ziele,
	description={
		Ziele definieren, was mit dem Produkt erreicht werden soll. Im Gegensatz zu Anforderungen geht es dabei weniger um die technischen Aspekte, als das zugrundeliegende Problem, dass mit dem Produkt gelöst werden soll.
	}
}